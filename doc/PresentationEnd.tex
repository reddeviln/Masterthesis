\documentclass[12pt,a4paper,handout]{beamer}

\usefonttheme{professionalfonts}
%\usepackage[ngerman]{babel}% deutsches Sprachpaket wird geladen
\usepackage[ngerman,english]{babel}% englisches Sprachpaket wird geladen
\usepackage{tabularx}
\usepackage{lmodern}% Für die Schrift
\usepackage[T1]{fontenc} % westeuropäische Codierung wird verlangt
\usepackage[utf8]{inputenc}% Umlaute werden erlaubt
\usetheme{Berlin}
%\usepackage{showkeys} % Labels anzeigen
\usepackage{amsmath} % Erweiterung für den Mathe-Satz
\usepackage{amssymb} % alle Zeichen aus msam und msmb werden dargestellt
\usepackage{framed}
\usepackage[german]{fancyref}
\usepackage{listings}
\usepackage{color}
\definecolor{mygreen}{RGB}{28,172,0} % Define colour
\definecolor{mylilas}{RGB}{170,55,241}
\usepackage{graphicx} % Graphiken und Bilder können eingebunden werden
\usepackage{multirow} % erlaubt in einer Spalte einer Tabelle die Felder in mehreren Zeilen zusammenzufassen
\usepackage{url} % Dient zur Auszeichnung von URLs; setzt die Adresse in Schreibmaschinenschrift.
\usepackage[center]{caption}  % Bildunterschrift wird zentriert
\usepackage{subfigure} % mehrere Bilder können in einer figure-Umgebung verwendet werden
\usepackage{longtable} % Diese Umgebung ist ähnlich definiert wie die tabular-Umgebung, erlaubt jedoch mehrseitige Tabellen.
\usepackage{amsthm} % erlaubt die Benutzung von eigenen Theoremen
\usepackage{hyperref} % Links und Verweise werden innerhalb von PDF Dokumenten erzeugt
\usepackage{wrapfig} % Das Paket ermöglicht es von Schrift umflossene Bilder und Tabellen einzufügen.
%\numberwithin{equation}{section} % Nummerierungen der Gleichungen, die durch equation erstellt werden, sind gebunden an die section
\usepackage{latexsym} % LaTeX-Symbole werden geladen
\usepackage{tikz} % Erlaubt es mit tikz zu zeichnen
\usepackage{tabularx} % Erlaubt Tabellen 
\usepackage{algorithm} % Erlaubt Pseudocode
\usepackage{algorithmic}
\usepackage{color} % Farbpaket wird geladen
\usepackage{stmaryrd} % St Mary Road Symbole werden geladen
\usepackage{csquotes}
\usepackage{bm}
\usepackage{todonotes}
\usepackage{lipsum}
\usepackage{multicol}
\usepackage{multimedia}

% Hier werden neue Theorems erstellt.
\theoremstyle{definition}
\newtheorem{auf}{Aufgabe}
\newtheorem{rem}[auf]{Remark}
\newtheorem{defn}[auf]{Definition}
\newtheorem{bsp}[auf]{Example}
\newtheorem{notation}[auf]{Notation}
\theoremstyle{plain}
\newtheorem{kor}[auf]{Corollary}
\newtheorem{sa}[auf]{Theorem}
\newtheorem{lem}[auf]{Lemma}
\newtheorem{alg}[auf]{Algorithm}
\DeclareMathOperator*{\esssup}{ess\,sup} % essentiellen Supremums
\DeclareMathOperator{\spn}{span} % Span
\DeclareMathOperator{\supp}{supp} % Träger
\DeclareMathOperator{\ddiv}{div} % divergenz
\newcommand{\abs}[1]{\left\vert #1\right\vert}
\newcommand{\dotp}[2]{\left\langle #1,#2\right\rangle}
\newcommand{\rr}{\mathbb{R}}
\newcommand{\g}{~\textgreater ~}
\newcommand{\ls}{~\textless ~}
\renewcommand{\algorithmicrequire}{\textbf{Input:}}
\renewcommand{\algorithmicensure}{\textbf{Output:}}
\newcommand{\cc}{\mathbb{C}}
\newcommand{\kk}{\mathbb{K}}
\newcommand{\nn}{\mathbb{N}}
\newcommand{\qq}{\mathbb{Q}}
\newcommand{\e}{\varepsilon\g 0~}
\newcommand{\fe}{\forall \e}
\newcommand{\so}{\sum_{k=0}^{n}}
\newcommand{\si}{\sum_{k=1}^{n}}
\newcommand{\soi}{\sum_{k=0}^{\infty}}
\newcommand{\sii}{\sum_{k=1}^{\infty}}
\newcommand{\de}{\mathrm{d}}
\newcommand{\norm}[1]{\left\lVert#1\right\rVert}
\newcommand{\lpnorm}[1]{\left(\int\abs{#1}^2\D\Omega \right)^{1/2}}
\newcommand{\bfu}{\bm{u}}
\newcommand{\bff}{\bm{f}}
\newcommand{\bfB}{\bm{B}}
\newcommand{\bfb}{\bm{b}}
\newcommand{\bfs}{\bm{s}}
\newcommand{\bfC}{\bm{C}}
\newcommand{\bfx}{\bm{x}}
\newcommand{\bfR}{\bm{R}}
\newcommand{\D}{\mathop{}\!\mathrm{d}}
\floatname{algorithm}{Algorithmus}
%\renewcommand{\thesubfigure}{the.g}
\addto{\captionsenglish}{\renewcommand{\bibname}{References}}
\lstset{language=Matlab,
    basicstyle={\scriptsize \ttfamily},
    breaklines=true,
    morekeywords={matlab2tikz},
    keywordstyle=\color{blue},
    morekeywords=[2]{1}, 
    keywordstyle=[2]{\color{black}},
    identifierstyle=\color{black},
    stringstyle=\color{mylilas},
    commentstyle=\color{mygreen},
    showstringspaces=false, %without this there will be a symbol in the places where there is a space
    numbers=left,
    numberstyle={\tiny \color{black}}, % size of the numbers
    numbersep=9pt, % this defines how far the numbers are from the text
    emph=[1]{for,end,break},
    emphstyle=[1]\color{red}, %some words to emphasise
}
\makeatletter
\renewcommand{\p@subfigure}{}
\renewcommand{\@thesubfigure}{\thesubfigure:\hskip\subfiglabelskip}
\makeatother
\begin{document}
    \title{Numerical Solution of Integro-Differential Equations}
    \author{Nils Dornbusch}
    \date{July 09, 2019}
    \maketitle
    \begin{frame}
    \tableofcontents[pausesections]
    \end{frame}
\section{Introduction}
\frame{
    \frametitle{Introduction}
    A Non-Newtonian fluid is a fluid that changes its viscosity under stress.
        There are many examples for them. These include
        \begin{itemize}[<+->]
        \item toothpaste
        \item liquid plastic
        \item granular flow
        \item and many more.
        \end{itemize}}
\begin{frame}
    \frametitle{Motivation}
    \begin{itemize}[<+->]
        \item Simulation of this particular fluid model does not exist 
        \item First step in trying to develop a stable numerical approach
        \item To develop a starting point for future projects
    \end{itemize}
\end{frame}
\section{Physics}
    \begin{frame}
        \frametitle{Physics}
        \begin{itemize}[<+->]
            \item  The physical model uses the well-known Navier Stokes equations
            \item The stress tensor is modeled using an integral model
        \end{itemize}
    \end{frame}

       \begin{frame}
        \frametitle{The incompressible Navier Stokes equations}
        \begin{align*}
        \frac{\partial \bfu}{\partial t}+(\bfu\cdot \nabla)\bfu &= \bff +\nabla\cdot\sigma +\mu_s\Delta\bfu-\nabla p,\\
        \ddiv(\bfu)&= 0,\\
        \sigma(t,\hat\bfx)&= \int_{-\infty}^t-\partial_{t'}\bfB(t,t',\hat\bfx)G(t,t')\D t',\\
        \partial_t \bfB &=- (\bfu\cdot\nabla)\bfB+(\nabla \bfu)\bfB+\bfB(\nabla\bfu)^T.
        \end{align*}
        with e.g. $G(t,t')=\mu_p\cdot e^{-(t-t')/\lambda}$, $\bfu$ the velocity vector, $\bff$ a generic source term, $\sigma$ the stress tensor, $\mu_s, \mu_p$ the solvent and polymer viscosity respectively, $p$ the pressure, $\lambda$ the relaxation time and $\bfB$ the Finger tensor.
     \end{frame}
 \begin{frame}
     \frametitle{Boundary conditions}
     We choose suitable initial and boundary conditions for the velocity. For $\bfB$ we use
     \begin{align*}
         \bfB(t,t,\hat\bfx)& = \bm{1},\\
         \bfB(0,t',\hat\bfx) &=\bm{1},
     \end{align*}
     which means that we assume a stress free start of the simulation.
 \end{frame}
\section{Transformation of the equation}
    \begin{frame}
        \frametitle{Assumptions for the dimension reduction}
        These are very complex equations. This work uses the following assumptions:
        \begin{itemize}[<+->]
            \item $\bfu= (0,0,u)^T$, $\bff=\bm{0}$, $\nabla p\equiv(0,0,\partial_3 p)$
            \item $\partial_3\bfB =\bm{0}$
         \end{itemize}
    \uncover<3->{This yields a 2D model for the cross sectional flow in a e.g. tube.}\\
    
    \only<4>{\begin{figure}
            \includegraphics[width=0.5\textwidth]{h006-laminar-flow}
            \caption{Visualization of the assumptions (reference in the thesis)}
         \end{figure}}
    \end{frame}
\begin{frame}
    \frametitle{Dimension reduction}
    These assumptions yield the 2D equations 
    \begin{align*}
    \partial_t u(t,\bfx) &= -\partial_3 p +\nabla\cdot \bfs+\mu_s\Delta u,\\
    \bfs(t,\bfx) &=\int_{-\infty}^t-\partial_{t'}\bfb(t,t',\bfx)G(t,t')\D t',\label{eq:s2D}\\
    \partial_t\bfb(t,t',\bfx)&=
    \begin{pmatrix}
    \partial_1 u(t,\bfx)\\\partial_2 u(t,\bfx)
    \end{pmatrix}=\nabla u.
    \end{align*}
\end{frame}
\begin{frame}
    \frametitle{Laplace transform}
    We want to eliminate the integral. Our solution: 
    \begin{enumerate}[<+->]
        \item introduce $\tau= t-t'$ (age variable)
        \item put this into the equations
        \item perform the Laplace transform of $\bfb$ with respect to $\tau$
    \end{enumerate}
\end{frame}
\begin{frame}
    \frametitle{Laplace transform}
    The equation for the stress tensor with $\tau$ reads
    \begin{equation*}
        \bfs(t,\bfx)=\int_0^\infty\partial_\tau\bfb(t,t-\tau,\bfx)G(t,t-\tau)\D\tau.
    \end{equation*}
   Using the chain rule the governing equation for the Finger tensor transforms to  
   \begin{equation*}
   \partial_t \bfb +\partial_\tau\bfb=\nabla u
   \end{equation*}
\end{frame}
\begin{frame}
    \frametitle{Laplace transform}
    Transform $\bfb\mapsto L_b:=\int_0^\infty \bfb(x,t,t-\tau)e^{-s\tau}\D \tau$. This yields
    \begin{equation*}
    \partial_tL_{\bfb}(t,\bfx,s) + L_{\partial_\tau\bfb}(t,\bfx,s) = \frac{1}{s}\nabla u.
    \end{equation*}
    Second part of the sum:
        \begin{align*}
        L_{\partial_\tau\bfb}(t,\bfx,s) &= \int_0^\infty\partial_\tau\bfb(t,t-\tau,\bfx)e^{-s\tau}\D\tau\\ &=\lim_{r\to\infty}\bfb(t,t-r,\bfx)e^{-sr}-\bfb(t,t,\bfx)e^{-s\cdot 0}\\&+s\int_0^\infty\bfb(t,t-\tau,\bfx)e^{-s\tau}\D\tau.\\
        &= -\bfb(t,t,\bfx) +sL_{\bfb}(t,\bfx, s)\\
        &= sL_{\bfb}(t,\bfx,s).
        \end{align*}
\end{frame}
\begin{frame}
    \frametitle{Laplace transform}
    Using $C_s:=sL_b(t,\bfx,s)$ we obtain
    \begin{equation*}
         \partial_t\bfC_s(t,\bfx)+s\bfC_s(t,\bfx)=\nabla u
    \end{equation*}
    Until now no assumptions on the function type of $G$ were made.
    If we set $s:=\frac{1}{\lambda}$ and use $G(t,t-\tau)=\mu_pe^{-\tau/\lambda}$, it follows
    \begin{equation*}
        \bfs(t,\bfx)=\mu_pC_{1/\lambda}(t,\bfx)
    \end{equation*}
\end{frame}
\begin{frame}
    \frametitle{The transformed 2D equations}
    \begin{align*}
        \partial_t u(t,\bfx) &= -\partial_3 p +\nabla\cdot \bfs+\mu_s\Delta u,\\
        \label{eq:transfeq2}
        \bfs(t,\bfx)&=\mu_p\bfC_{1/\lambda},\\
        \partial_t\bfC_{1/\lambda}(t,\bfx) &= -\frac{1}{\lambda}\bfC_{1/\lambda}(t,\bfx)+\frac{1}{\lambda}\nabla u    \end{align*}
        This is a linear system of which the existence and uniqueness of the solution is proven in the thesis.
\end{frame}
\section{Numerics}
\begin{frame}
    \frametitle{The software}
    The software used is FEniCS. 
    \begin{itemize}[<+->]
        \item Framework for solving PDEs with finite element discretization (limited DG support available)
        \item have to plug in the weak form and maybe time loop
        \item MPI support, however not tested yet 
    \end{itemize}
    \begin{figure}
        \includegraphics[width=0.5\textwidth]{FenicsCode}
        \caption{Weak formulation for the derived equations with timestepping}
    \end{figure}
\end{frame}
\begin{frame}
\frametitle{Simulation cases}
The following cases were simulated
\begin{itemize}[<+->]
    \item Circle mesh 
    \begin{itemize}[<+->]
        \item Startup flow with a fixed pressure 
        \item Flow with manufactured solution for convergence
    \end{itemize}
    \item square mesh (side length = 2)
    \begin{itemize}[<+->]
        \item cross section of an ideal rheometer
    \end{itemize}
\end{itemize}
\uncover<6>{Note: Units have been omitted but have to be chosen consistently. }
\end{frame}
\begin{frame}
    \frametitle{Numerical results}
    Two domains: circle and square
    \begin{figure}
        \caption{Meshes with }
        \subfigure[402987 cells]{\includegraphics[width=0.4\textwidth]{MeshCircle}}
        \subfigure[4852 cells]{\includegraphics[width=0.4\textwidth]{MeshSquare}}
    \end{figure}
\end{frame}

\begin{frame}
    \frametitle{Startup flow fixed pressure}
    \begin{figure}
        \subfigure{\includegraphics[width=0.4\textwidth]{vel}}
        \subfigure{\includegraphics[width=0.5\textwidth]{centerlinevel}}
        \caption{Startup flow Maxwell with $E=1$ and $p=-3$}
    \end{figure}
\end{frame}
\begin{frame}
    \frametitle{Numerical results}
    Convergence in the $L^2$-norm was achieved with a timestep width of $0.1$ and an end time of $1$.
    \begin{table}
        \centering
        \begin{tabular}{c|c|c}
            $\approx$ \# Elements per diameter& $\mathrm{L}^2$-error&EOC\\
            \hline
            20 & 0.0797652 & -\\
            40 & 0.0233543 & 1.77207\\
            80 & 0.00615334 & 1.92425\\
            160 & 0.00163421 & 1.91278\\
            320 & 0.000433982 & 1.91288
        \end{tabular}
        \caption{Convergence of the \textsc{Maxwell} model $(\mu_s=0)$}
    \end{table}
\end{frame}
\begin{frame}
    \frametitle{Cross section of an ideal rheometer}
    
\end{frame}
\section{Other approaches for the integral eq.}
\begin{frame}
\frametitle{Different approaches for the integral equation}
    What happens if the \enquote{\textsc{Laplace} trick} is not possible? 
    \begin{itemize}[<+->]
        \item look at a combination of multiple relaxation times
        \item brute force (mainly as reference)
        \item exponentially increasing time intervals
        \item transformation of the argument
    \end{itemize}
\end{frame}
\begin{frame}
    \frametitle{Combination of relaxation times}
    Let $(\lambda_i)_{i=1,\dotsc,N},(\mu_p^i)_{i=1,\dotsc,N}$ be a sequence of relaxation times and polymer viscosities respectively. We set
    \begin{equation*}
    G(\tau)=\sum_{n=1}^{N}\mu_p^{(n)}e^{-\tau/\lambda_n},
    \end{equation*}
\end{frame}
\begin{frame}
    \frametitle{Combination of relaxation times}
    Putting $G$ into the governing equation yields
    \begin{equation*}
    \bfs(t) = \sum_{n=1}^{N}\mu_p^{(n)}\int_0^t\partial_\tau \bfb(t,t-\tau)e^{-\tau/\lambda_n}\D\tau.
    \end{equation*}
    Using similar arguments as before results in 
    \begin{equation}
    \bfs(t)=\sum_{n=1}^N\frac{\mu_p^{(n)}}{\lambda_n}L_{\bfb}(t,1/\lambda_n)=\sum_{n=1}^N\mu_p^{(n)}\bfC_{1/\lambda_n}.
    \end{equation}
\end{frame}
\begin{frame}
    \frametitle{Combination of relaxation times}
    How do we interpret the result?
    \begin{itemize}[<+->]
        \item For each relaxation time we obtain
        \begin{equation*}
            \partial_t \bfC_{1/\lambda_n}+\frac{1}{\lambda_n}\bfC_{1/\lambda_n}=\nabla u
        \end{equation*}
        \item more equations $\Rightarrow$ more computational effort
        \item we retain existence and uniqueness of solution
    \end{itemize}
\end{frame}
\begin{frame}
    \frametitle{Introducing a new equation}
    Remaining two approaches considered on
    \begin{align*}
    \dot\phi(t)&=-\phi(t)-\int_0^tm(t-\tau)\dot\phi(\tau)\D\tau,\\
    m(t)&=v_1\phi(t)+v_2\phi(t)^2,
    \end{align*}
    with $v_1,v_2\in\rr$ parameters. Why do we use these instead?
    \begin{itemize}[<+->]
        \item fewer dimensions and equations
        \item same difficulty regarding the integral
        \item easier to test methods
    \end{itemize}
\end{frame}
\begin{frame}
    \frametitle{Introducing a new equation}
    We can transform the previous equation to 
    \begin{equation*}
        \phi(t)=\phi(0)+\int_0^t[m(\tau)(\phi(0)-\phi(t-\tau))-\phi(\tau)]\D \tau.
    \end{equation*}
    We discretize $t$ with $0=t_0,\dotsc,t_N=T$, where $T$ is the end time.
\end{frame}
\begin{frame}
    \frametitle{Brute force algorithm}
    By using equal distant points $t_j$ and the trapezoidal rule we obtain the brute force algorithm. Drawbacks:
    \begin{itemize}[<+->]
        \item computational effort $\mathcal{O}(N^2)$
        \item the calculation of each time step $t_{n+1}$ is more expensive than $t_{n}$ for all $n$
    \end{itemize}
    We use it to verify the other ideas.
\end{frame}
\begin{frame}
    \frametitle{Exponentially increasing time intervals}
    Idea behind this approach:
    \begin{itemize}[<+->]
        \item $\phi$ is exponentially declining in $t$
        \item similar trick used in Fast-\textsc{Fourier}-Transformation
    \end{itemize}
We set $t_j=h2^j$ for $j=1,\dotsc N$ and use the quadrature rule
\begin{equation*}
    \int_0^{t_n}f(t)dt=\sum_{j=0}^{n-1}\int_{t_j}^{t_{j+1}}f(t)\D t\approx\sum_{j=0}^{n-1}(t_{j+1}-t_j)f_{j+1}
\end{equation*}
for a generic function $f$.
\end{frame}
\begin{frame}
    \frametitle{Exponentially increasing time intervals}
    Using the quadrature rule yields
    \begin{equation*}
        \phi_n = \phi_0 + \sum_{j=0}^{n-1}h_j[m_{j+1}(\phi_0-\phi(t_n-t_{j+1}))-\phi_{j+1}].
    \end{equation*}
    The update from $\phi_n$ to $\phi_{n+1}$ is given by
    \begin{equation*}
        \phi_{n+1}= \phi_n -h_n\phi_{n+1}+\sum_{j=0}^{n-1}h_jm_{j+1}(\phi(t_n-t_{j+1})-\phi(t_{n+1}-t_{j+1})).
    \end{equation*}
\end{frame}
\begin{frame}
    \frametitle{Exponentially increasing time intervals}
    We have the problem that $t_n-t_{j+1}$ and $t_{n+1}-t_{j+1}$ are not equal to $t_j$ for some $j$ in general. The solution is interpolating $t_n-t_{j+1}\approx t_n$ and $t_{n+1}-t_{j+1}\approx t_{n+1}$. After reordering we obtain
    \begin{equation*}
         \phi_{n+1}(1 + h_n + \sum_{j=0}^{n-2}h_jm_{j+1})=\phi_n(1+\sum_{j=0}^{n-2}h_jm_{j+1})+h_{n-1}m_n(\phi_0-\phi_n).
    \end{equation*}
\end{frame}
\begin{frame}
    \frametitle{Exponentially increasing time intervals}
    Now we have to take care of the sums. We multiply both sides by $2^{-n}$ and set $\beta_n=2^{-n}\sum_{j=0}^{n-2}h_jm_{j+1}$. It yields
    \begin{align*}
    \phi_{n+1}(2^{-n}+h+\beta_n) &= \phi_n(2^{-n}+\beta-\frac{h}{2}m_n)+\frac{h}{2}m_n\phi_0,\\
    \beta_{n+1} &= \frac{1}{2}(\beta_n+\frac{h}{2}m_n).
    \end{align*}
\end{frame}
\begin{frame}
    \frametitle{Analysis of exp. increasing time intervals}
    How do we interpret the results?
    \begin{itemize}[<+->]
        \item computational effort $\mathcal{O}(N)$
        \item unclear if $h\to 0$ converges to the solution
        \item error to great (interpolation to bad?)
        \begin{figure}
            \centering
            \includegraphics[width=0.4\textwidth]{Phidiff}
            \caption{Both $\phi$ with $h=\Delta t=0.01$ and $v_1=1.5,~v_2=0.5$}
        \end{figure}
    \end{itemize}
\end{frame}
\begin{frame}
    \frametitle{Transformation of the argument}
    Idea for this approach:
    \begin{itemize}
        \item transform $t=f(u)$ to obtain better properties for the integral
    \end{itemize}
    Inserting this transformation and setting $\tilde\phi(u)=\phi(f(u))$ yields
    \begin{align*}
        \tilde{\phi}(u)&=\tilde{\phi}(u_{-1}) +\int_{u_{-1}}^{u}f'(u')[\tilde{m}(u')(\tilde\phi(u_{-1})\\&-\tilde{\phi}(f^{-1}(f(u)-f(u'))))-\tilde{\phi}(u')]\D u',
    \end{align*}
    where $u_{-1}=f^{-1}(0)$.
\end{frame}
\begin{frame}
    \frametitle{Transformation of the argument}
    The problematic part is
    \begin{equation*}
        \int_{u_{-1}}^{u}f'(u')\tilde{m}(u')\tilde{\phi}(f^{-1}(f(u)-f(u')))\D u'.
    \end{equation*}
    Ideally,
    \begin{equation*}
        F(u,u'):=f^{-1}(f(u)-f(u'))=u
    \end{equation*}
    almost everywhere.
\end{frame}
\begin{frame}
    \frametitle{Transformation of the argument}
    \begin{figure}
        \centering
        \includegraphics[width=0.8\textwidth]{HistoryF}
        \caption{History dependency for different $f$}
    \end{figure}
\end{frame}
\begin{frame}
    \frametitle{Transformation of the argument}
    Interpretation of this figure:
    \begin{itemize}[<+->]
        \item Both have a sharp drop in the end
        \item $f(u)=\exp(\pi/2\sinh(u))$ is better suited in this regard
        \item but it results in a qualitatively different curve for each $u$
        \item we take $f(u)=\exp(u)$ because of this 
    \end{itemize}
\end{frame}
\begin{frame}
    \frametitle{Transformation of the argument}
    We only discuss $\int_{u_{-1}}^{u}f'(u')\tilde{m}(u')\tilde{\phi}(f^{-1}(f(u)-f(u')))\D u'$ here. We use $u_j=u_0+jh$ with $j=0,\dotsc,N$ and $\abs{f(u_0)-f(u_{-1})}$ sufficiently small. We obtain
    \begin{equation*}
        I_2:=\int_{u_{-1}}^{u_{n+1}}f'(v)\tilde{m}(v)\tilde{\phi}(F(u_{n+1},v))\D v
    \end{equation*}
\end{frame}
\begin{frame}
    \frametitle{Transformation of the argument}
    The big question is: Which integration points do we use for $I_2$? Observe that
    \begin{equation*}
        w=F(u,u')\Rightarrow u'=F(u,w).
    \end{equation*}
    We define 
    \begin{equation*}
        w_{i}^{n}:=F(u_n,u_i).
    \end{equation*}
    Then we use the discretization 
    \begin{equation*}
        (v_i)_{i=0,\dotsc,2n+1}=\text{sorted}\{(u_i)_{i=0,\dotsc,n}\cup (w^n_{i})_{i=0,\dotsc,n-1}\}
    \end{equation*}
\end{frame}
\begin{frame}
    \frametitle{Transformation of the argument}
    This is possible because we assume
    \begin{equation*}
        \tilde\phi(v)=\tilde{\phi}(u_i)\quad\forall v\in (u_{i-1},u_i]
    \end{equation*}
    for all $i=0,\dotsc,N$ so we use constant ansatz functions on each interval.
\end{frame}
\begin{frame}
\frametitle{Transformation of the argument}
    \begin{figure}
        \centering
        \includegraphics[width=\textwidth]{Tranformls1}
        \caption{Visualization of discretization for $I_2$, where $n=3$}
        \label{fig:vis1}
    \end{figure}
\end{frame}
\begin{frame}
\frametitle{Transformation of the argument}
    Important to know: How many intervals do the $w$ spread? So find $n+1-j$ such that 
    \begin{equation*}
        w^{n+1}_i>u_j \quad\forall i=0\dotsc,n.
    \end{equation*}
    This can be calculated as
    \begin{equation*}
    l^*:=\left\lceil  \frac{u_n -w_{n}^{n+1}}{h}\right\rceil.
    \end{equation*}
    Is independent of $n$ because of the choice for $f$.
\end{frame}
\begin{frame}
    \frametitle{Transformation of the argument}
    What are the next steps?
    \begin{itemize}
        \item Divide $I_2$ into the different intervals
        \item the sorting of $(v_i)$ can be done it advance
        \item only shifting necessary in each timestep
    \end{itemize}
\end{frame}
\begin{frame}
    \frametitle{Transformation of the argument}
    \begin{table}
        \scriptsize
        \begin{tabular}{c|c|c|c|c}
            \# Elements & $\mathrm{L}^2$-error&EOC in $\mathrm{L}^2$&$\mathrm{L}^\infty$-error &EOC in $\mathrm{L}^\infty$\\
            \hline
            50 & 1.29927 & - & 0.0250247  & -\\
            100 & 0.450672 & 1.5275475676992607 & 0.0101168 & 1.3066064275748301\\
            200 & 0.153403 & 1.5547536813705434 & 0.00407696 &1.311182153107891\\
            400 & 0.0558441 & 1.4578457386762156 & 0.0017411 &1.2274971804453685\\
            800 & 0.0222726 & 1.3261379803853823 & 0.00147244  &0.2417923432251396
        \end{tabular}
        \caption{Convergence of the new algorithm for the simpler integral equation}
    \end{table}
\end{frame}
\begin{frame}
    \frametitle{Transformation of the argument}
    Everything works as it should. One might be tempted to think of effort $\mathcal{O}(N)$ but
    \begin{figure}
        \centering
        \includegraphics[width=0.5\textwidth]{runtime}
        \caption{Relationship of $l^*$ and $N$}
    \end{figure}
\end{frame}
\begin{frame}
    \frametitle{Transformation of the argument}
    Main takeaways:
    \begin{itemize}
       \item Unfortunately $\mathcal{O}(N^2)$ runtime
       \item but every timestep is equally fast
       \item good starting point for further studies
    \end{itemize}
\end{frame}
\section{Conclusion and outlook}
\begin{frame}
    \frametitle{Conclusion}
    \begin{itemize}[<+->]
        \item Simulation of a 2D problem 
        \item theoretical existence and uniqueness were proven
        \item results match physical expectations
        \item numerical convergence was obtained in practice
        \item simulation of a flow inside a rheometer cross section
        \item good starting for the integral without \enquote{\textsc{Laplace}} trick
    \end{itemize}
\end{frame}
\begin{frame}
    \frametitle{Outlook}
    Open questions:
    \begin{itemize}[<+->]
          \item Can the newly developed algorithm be altered to obtain runtime between $\mathcal{O}(N)$ and $\mathcal{O}(N^2)$?
          \item What if the flow is not only limited to one direction? 
          \item How can parallelization speed up the calculation? 
          \item How does one handle a continuous set of relaxation times?
          \item How can one use a simulation to perform parameter estimations?     
    \end{itemize}
\end{frame}
\begin{frame}
\huge{Thank you for your attention!}
\end{frame}
\end{document}

%%%%%%%%%%%%%%%%%%%%%%%%%%%%%%%%%%%%%%%%%%%%%%%%%%%%%%%%%%%%%%%%%%%%%%
% LaTeX Vorlage: Mathematische Texte
%
% Quelle: http://www.mi.uni-koeln.de/wp-MIEDV/% Datum: Juli 201% Copyright Universität zu Köln
% 
%%%%%%%%%%%%%%%%%%%%%%%%%%%%%%%%%%%%%%%%%%%%%%%%%%%%%%%%%%%%%%%%%%%%%%
\documentclass[12pt,a4paper]{scrartcl}

\addtokomafont{sectioning}{\rmfamily}
%\usepackage[ngerman]{babel}% deutsches Sprachpaket wird geladen
\usepackage[english]{babel}% englisches Sprachpaket wird geladen
\usepackage{tabularx}
\usepackage[T1]{fontenc} % westeuropäische Codierung wird verlangt
\usepackage[utf8]{inputenc}% Umlaute werden erlaubt
\usepackage[usenames]{color} % Erlaubt die Benutzung der namen im Farbpaket und deren Änderung
%\usepackage{showkeys} % Labels anzeigen
\usepackage{amsmath} % Erweiterung für den Mathe-Satz
\usepackage{amssymb} % alle Zeichen aus msam und msmb werden dargestellt
\usepackage{framed}
\usepackage[german]{fancyref}
\usepackage{listings}
\usepackage{color}
\definecolor{mygreen}{RGB}{28,172,0} % Define colour
\definecolor{mylilas}{RGB}{170,55,241}
\usepackage{graphicx} % Graphiken und Bilder können eingebunden werden
\usepackage{multirow} % erlaubt in einer Spalte einer Tabelle die Felder in mehreren Zeilen zusammenzufassen
\usepackage{enumerate} % erlaubt Nummerierungen 
\usepackage{url} % Dient zur Auszeichnung von URLs; setzt die Adresse in Schreibmaschinenschrift.
\usepackage[center]{caption}  % Bildunterschrift wird zentriert
\usepackage{subfigure} % mehrere Bilder können in einer figure-Umgebung verwendet werden
\usepackage{longtable} % Diese Umgebung ist ähnlich definiert wie die tabular-Umgebung, erlaubt jedoch mehrseitige Tabellen.
\usepackage{paralist} % Modifikation der bereits bestehenden Listenumgebungen
\usepackage{lmodern}% Für die Schrift
\usepackage{amsthm} % erlaubt die Benutzung von eigenen Theoremen
\usepackage{hyperref} % Links und Verweise werden innerhalb von PDF Dokumenten erzeugt
\usepackage{wrapfig} % Das Paket ermöglicht es von Schrift umflossene Bilder und Tabellen einzufügen.
\numberwithin{equation}{section} % Nummerierungen der Gleichungen, die durch equation erstellt werden, sind gebunden an die section
\usepackage{latexsym} % LaTeX-Symbole werden geladen
\usepackage{tikz} % Erlaubt es mit tikz zu zeichnen
\usepackage{tabularx} % Erlaubt Tabellen 
\usepackage{algorithm} % Erlaubt Pseudocode
\usepackage{algorithmic}
\usepackage{color} % Farbpaket wird geladen
\usepackage{stmaryrd} % St Mary Road Symbole werden geladen
\usepackage{csquotes}

% Hier werden neue Theorems erstellt.
\theoremstyle{definition}
\newtheorem{auf}{Aufgabe}
\newtheorem{rem}[auf]{Bemerkung}
\newtheorem{defn}[auf]{Definition}
\newtheorem{bsp}[auf]{Beispiel}
\theoremstyle{plain}
\newtheorem{kor}[auf]{Korollar}
\newtheorem{sa}[auf]{Satz}
\newtheorem{alg}[auf]{Algorithmus}
\DeclareMathOperator*{\esssup}{ess\,sup} % essentiellen Supremums
\DeclareMathOperator{\spn}{span} % Span
\DeclareMathOperator{\supp}{supp} % Träger
\newcommand{\abs}[1]{\left\vert #1\right\vert}
\newcommand{\rr}{\mathbb{R}}
\newcommand{\g}{~\textgreater ~}
\newcommand{\ls}{~\textless ~}
\renewcommand{\algorithmicrequire}{\textbf{Input:}}
\renewcommand{\algorithmicensure}{\textbf{Output:}}
\newcommand{\cc}{\mathbb{C}}
\newcommand{\e}{\varepsilon\g 0~}
\newcommand{\fe}{\forall \e}
\newcommand{\so}{\sum_{k=0}^{n}}
\newcommand{\si}{\sum_{k=1}^{n}}
\newcommand{\soi}{\sum_{k=0}^{\infty}}
\newcommand{\sii}{\sum_{k=1}^{\infty}}
\newcommand{\de}{\mathrm{d}}
\newcommand{\norm}[1]{\left\lVert#1\right\rVert}
\floatname{algorithm}{Algorithmus}
\renewcommand{\thesubfigure}{\thefigure.\arabic{subfigure}}
\lstset{language=Matlab,
    basicstyle={\scriptsize \ttfamily},
    breaklines=true,
    morekeywords={matlab2tikz},
    keywordstyle=\color{blue},
    morekeywords=[2]{1}, 
    keywordstyle=[2]{\color{black}},
    identifierstyle=\color{black},
    stringstyle=\color{mylilas},
    commentstyle=\color{mygreen},
    showstringspaces=false, %without this there will be a symbol in the places where there is a space
    numbers=left,
    numberstyle={\tiny \color{black}}, % size of the numbers
    numbersep=9pt, % this defines how far the numbers are from the text
    emph=[1]{for,end,break},
    emphstyle=[1]\color{red}, %some words to emphasise
}
\makeatletter
\renewcommand{\p@subfigure}{}
\renewcommand{\@thesubfigure}{\thesubfigure:\hskip\subfiglabelskip}
\makeatother
\begin{document}
% Hier wird die Titelseite erstellt
\begin{titlepage}
\pagestyle{empty}
\begin{center}
\newcommand{\HRule}{\rule{\linewidth}{0.7mm}}
\textsc{\LARGE University of Cologne }\\ [0.4cm]
\textsc{ Department of Mathematics} \\[1.5cm]
\includegraphics[width=0.45\textwidth]{uni}\\[1.5cm]  % Uni-Logo wird geladen
\HRule \\[0.4cm]
{ \huge \bfseries Numerical Solution of Integro-Differential equations}\\[0.4cm]
\HRule \\[1cm]
\textsc{\Large Master's thesis}\\[2mm]
\textsc{\today}\\[10mm]
\textsc{in cooperation with the German Aerospace Center}\\[1.0cm]
\includegraphics[width=0.45\textwidth]{DLR-Logo-full}\\[1.0cm]


  




\begin{center}

\textsc{\Large Nils Dornbusch} \\[3pt]
\textsc{\Large first examiner: Prof. Dr-Ing. Gassner}
\end{center}
\end{center}
\end{titlepage}
\section*{Danksagung}
Ich möchte mich bei Herrn Professor Gassner und seiner Arbeitsgruppe ganz herzlich für die kompetente und umfangreiche Betreuung bedanken. Insbesondere bei Herrn Lucas Friedrich für sein außerordentliches Engagement, das weit über Sprechstundenzeiten hinaus ging. 
\par Zusätzlich bedanke ich mich noch bei meiner Familie, meinen Freunden und Kommilitonen und meiner Freundin für ihre Geduld und ihr Verständnis.
\\[1cm]
Köln, \today 
\\[1cm]
Nils Dornbusch
\newpage
\tableofcontents
\newpage
\section{Introduction}
This work aims to provides an analysis of the Integro-Differential-equations which occur when modeling the behaviour of non-Newtonian fluids. The motivation arises in Physics when studying for example liquid glass or granule. This is often needed in space applications. I will reference the relevant physical phenomena which the interested reader can follow to learn more about as this is not the main focus of this work.\\
The main focus will actually lie in the numerical challenges, which the equations present. We will also touch upon issues regarding the implementation.
\par 
The biggest problem when trying to implement these equation lies in the Integral over older timesteps and the appearence of another time dimension, which we call the \enquote{age dimension}. But more on that later on.\\
We will take the following steps to study said equations:
\begin{itemize}
    \item theoretical derivation of the Navier-Stokes-Equations for the non-Newtonian stress tensor
    \item discussion of different spatial and time discretization methods
    \item challenges in the implementation
    \item some application with different complexity 
\end{itemize}
For the sake of simplicity we will use well known framework FEniCS \cite{Alnaes2012a}\cite{AlnaesBlechta2015a}\cite{AlnaesEtAl2012}\cite{AlnaesLoggEtAl2012a} which uses the underlying DOLFIN library \footnote{citation needed} and we will make assumptions that reduce the complexity of the problem \par 
At the end of this work we would like to answer the question: \enquote{Is this numerical method worth to try with the full system in 5-dimensions (3 spatial + 1 time  + 1 age dimension) in exa-scale systems?}\par 
So without further ado let us dive into the matter!
\newpage
\addcontentsline{toc}{section}{List of figures}
\setcounter{lofdepth}{2}
\listoffigures
\newpage
\addcontentsline{toc}{section}{References}
\bibliographystyle{amsplain}
\bibliography{References}{}
\newpage
\section*{Eigenständigkeitserklärung}
Hiermit versichere ich, Nils Dornbusch, an Eides statt, dass ich die vorliegende Arbeit selbstständig und ohne die Benutzung anderer als der angegebenen Hilfsmittel angefertigt habe. Alle Stellen, die wörtlich oder sinngemäß aus veröffentlichten und nicht veröffentlichten Schriften entnommen wurden, sind als solche kenntlich gemacht. Die Arbeit ist in gleicher oder ähnlicher Form oder auszugsweise im Rahmen einer anderen Prüfung noch nicht vorgelegt worden. Ich versichere, dass die eingereichte elektronische Fassung der eingereichten Druckfassung vollständig entspricht.
\\[\bigskipamount]
Köln, \today
\\[2\bigskipamount]
Nils Dornbusch
\end{document}

%%%%%%%%%%%%%%%%%%%%%%%%%%%%%%%%%%%%%%%%%%%%%%%%%%%%%%%%%%%%%%%%%%%%%%
% LaTeX Vorlage: Mathematische Texte
%
% Quelle: http://www.mi.uni-koeln.de/wp-MIEDV/% Datum: Juli 201% Copyright Universität zu Köln
% 
%%%%%%%%%%%%%%%%%%%%%%%%%%%%%%%%%%%%%%%%%%%%%%%%%%%%%%%%%%%%%%%%%%%%%%
\documentclass[12pt,a4paper]{scrartcl}

\addtokomafont{sectioning}{\rmfamily}
%\usepackage[ngerman]{babel}% deutsches Sprachpaket wird geladen
\usepackage[english]{babel}% englisches Sprachpaket wird geladen
\usepackage{tabularx}
%\usepackage{lmodern}% Für die Schrift
\usepackage[T1]{fontenc} % westeuropäische Codierung wird verlangt
\usepackage[utf8]{inputenc}% Umlaute werden erlaubt
\usepackage[usenames]{color} % Erlaubt die Benutzung der namen im Farbpaket und deren Änderung
%\usepackage{showkeys} % Labels anzeigen
\usepackage{amsmath} % Erweiterung für den Mathe-Satz
\usepackage{amssymb} % alle Zeichen aus msam und msmb werden dargestellt
\usepackage{framed}
\usepackage[german]{fancyref}
\usepackage{listings}
\usepackage{color}
\definecolor{mygreen}{RGB}{28,172,0} % Define colour
\definecolor{mylilas}{RGB}{170,55,241}
\usepackage{graphicx} % Graphiken und Bilder können eingebunden werden
\usepackage{multirow} % erlaubt in einer Spalte einer Tabelle die Felder in mehreren Zeilen zusammenzufassen
\usepackage{enumerate} % erlaubt Nummerierungen 
\usepackage{url} % Dient zur Auszeichnung von URLs; setzt die Adresse in Schreibmaschinenschrift.
\usepackage[center]{caption}  % Bildunterschrift wird zentriert
\usepackage{subfigure} % mehrere Bilder können in einer figure-Umgebung verwendet werden
\usepackage{longtable} % Diese Umgebung ist ähnlich definiert wie die tabular-Umgebung, erlaubt jedoch mehrseitige Tabellen.
\usepackage{paralist} % Modifikation der bereits bestehenden Listenumgebungen
\usepackage{amsthm} % erlaubt die Benutzung von eigenen Theoremen
\usepackage{hyperref} % Links und Verweise werden innerhalb von PDF Dokumenten erzeugt
\usepackage{wrapfig} % Das Paket ermöglicht es von Schrift umflossene Bilder und Tabellen einzufügen.
\numberwithin{equation}{section} % Nummerierungen der Gleichungen, die durch equation erstellt werden, sind gebunden an die section
\usepackage{latexsym} % LaTeX-Symbole werden geladen
\usepackage{tikz} % Erlaubt es mit tikz zu zeichnen
\usepackage{tabularx} % Erlaubt Tabellen 
\usepackage{algorithm} % Erlaubt Pseudocode
\usepackage{algorithmic}
\usepackage{color} % Farbpaket wird geladen
\usepackage{stmaryrd} % St Mary Road Symbole werden geladen
\usepackage{csquotes}
\usepackage{bm}

% Hier werden neue Theorems erstellt.
\theoremstyle{definition}
\newtheorem{auf}{Aufgabe}
\newtheorem{rem}[auf]{Bemerkung}
\newtheorem{defn}[auf]{Definition}
\newtheorem{bsp}[auf]{Beispiel}
\theoremstyle{plain}
\newtheorem{kor}[auf]{Korollar}
\newtheorem{sa}[auf]{Satz}
\newtheorem{alg}[auf]{Algorithmus}
\DeclareMathOperator*{\esssup}{ess\,sup} % essentiellen Supremums
\DeclareMathOperator{\spn}{span} % Span
\DeclareMathOperator{\supp}{supp} % Träger
\DeclareMathOperator{\ddiv}{div} % divergenz
\newcommand{\abs}[1]{\left\vert #1\right\vert}
\newcommand{\rr}{\mathbb{R}}
\newcommand{\g}{~\textgreater ~}
\newcommand{\ls}{~\textless ~}
\renewcommand{\algorithmicrequire}{\textbf{Input:}}
\renewcommand{\algorithmicensure}{\textbf{Output:}}
\newcommand{\cc}{\mathbb{C}}
\newcommand{\e}{\varepsilon\g 0~}
\newcommand{\fe}{\forall \e}
\newcommand{\so}{\sum_{k=0}^{n}}
\newcommand{\si}{\sum_{k=1}^{n}}
\newcommand{\soi}{\sum_{k=0}^{\infty}}
\newcommand{\sii}{\sum_{k=1}^{\infty}}
\newcommand{\de}{\mathrm{d}}
\newcommand{\norm}[1]{\left\lVert#1\right\rVert}
\newcommand{\bfu}{\bm{u}}
\newcommand{\bff}{\bm{f}}
\newcommand{\bfB}{\bm{B}}
\newcommand{\bfb}{\bm{b}}
\newcommand{\bfs}{\bm{s}}
\newcommand{\bfC}{\bm{C}}
\newcommand{\bfx}{\bm{x}}
\newcommand{\D}{\mathop{}\!\mathrm{d}}
\floatname{algorithm}{Algorithmus}
\renewcommand{\thesubfigure}{\thefigure.\arabic{subfigure}}
\lstset{language=Matlab,
    basicstyle={\scriptsize \ttfamily},
    breaklines=true,
    morekeywords={matlab2tikz},
    keywordstyle=\color{blue},
    morekeywords=[2]{1}, 
    keywordstyle=[2]{\color{black}},
    identifierstyle=\color{black},
    stringstyle=\color{mylilas},
    commentstyle=\color{mygreen},
    showstringspaces=false, %without this there will be a symbol in the places where there is a space
    numbers=left,
    numberstyle={\tiny \color{black}}, % size of the numbers
    numbersep=9pt, % this defines how far the numbers are from the text
    emph=[1]{for,end,break},
    emphstyle=[1]\color{red}, %some words to emphasise
}
\makeatletter
\renewcommand{\p@subfigure}{}
\renewcommand{\@thesubfigure}{\thesubfigure:\hskip\subfiglabelskip}
\makeatother
\begin{document}
% Hier wird die Titelseite erstellt
\begin{titlepage}
\pagestyle{empty}
\begin{center}
\newcommand{\HRule}{\rule{\linewidth}{0.7mm}}
\textsc{\LARGE University of Cologne }\\ [0.4cm]
\textsc{ Department of Mathematics} \\[1.5cm]
\includegraphics[width=0.45\textwidth]{uni}\\[1.5cm]  % Uni-Logo wird geladen
\HRule \\[0.4cm]
{ \huge \bfseries Numerical Solution of Integro-Differential equations}\\[0.4cm]
\HRule \\[1cm]
\textsc{\Large Master's thesis}\\[2mm]
\textsc{\today}\\[10mm]
\textsc{in cooperation with the German Aerospace Center}\\[1.0cm]
\includegraphics[width=0.45\textwidth]{DLR-Logo-full}\\[1.0cm]


  




\begin{center}

\textsc{\Large Nils Dornbusch} \\[3pt]
\textsc{\Large first examiner: Prof. Dr-Ing. Gassner}
\end{center}
\end{center}
\end{titlepage}
\section*{Danksagung}
Ich möchte mich bei Herrn Professor Gassner und seiner Arbeitsgruppe ganz herzlich für die kompetente und umfangreiche Betreuung bedanken. Insbesondere bei Herrn Lucas Friedrich für sein außerordentliches Engagement, das weit über Sprechstundenzeiten hinaus ging. 
\par Zusätzlich bedanke ich mich noch bei meiner Familie, meinen Freunden und Kommilitonen und meiner Freundin für ihre Geduld und ihr Verständnis.
\\[1cm]
Köln, \today 
\\[1cm]
Nils Dornbusch
\newpage
\tableofcontents
\newpage
\section{Introduction}
This work aims to provides an analysis of the Integro-Differential-equations which occur when modeling the behaviour of non-Newtonian fluids. The motivation arises in Physics when studying for example liquid glass or granule. This is often needed in space applications. I will reference the relevant physical phenomena which the interested reader can follow to learn more about as this is not the main focus of this work.\\
The main focus will actually lie in the numerical challenges, which the equations present. We will also touch upon issues regarding the implementation.
\par 
The biggest problem when trying to implement these equation lies in the Integral over older timesteps and the appearence of another time dimension, which we call the \enquote{age dimension}. But more on that later on.\\
We will take the following steps to study said equations:
\begin{itemize}
    \item theoretical derivation of the Navier-Stokes-Equations for the non-Newtonian stress tensor
    \item discussion of different spatial and time discretization methods
    \item challenges in the implementation
    \item some application with different complexity 
\end{itemize}
For the sake of simplicity we will use the well known framework FEniCS \cite{Alnaes2012a}\cite{AlnaesBlechta2015a}\cite{AlnaesEtAl2012}\cite{AlnaesLoggEtAl2012a} which uses the underlying DOLFIN library \footnote{citation needed} and we will make assumptions that reduce the complexity of the problem. \par 
At the end of this work we would like to answer the question: \enquote{Is this numerical method worth to try with the full system in 5-dimensions (3 spatial + 1 time  + 1 age dimension) in exa-scale systems?}\par 
So without further ado let us dive into the matter!
\newpage
\section{Mathematical basics}
In this chapter various typical definitions and theorems are presented which will be used extensively throughout this work.
\subsection{Functional analysis}
\subsection{The finite element method}
\subsection{\textsc{Laplace} transform}
\section{Modeling}
\subsection{\textsc{Navier-Stokes} Equation}
\subsection{\textsc{Maxwell/Oldroyd-B} model}
\subsection{Dimension reduction}
After we have discussed a bit of physics in the previous sections, we will now focus on making these equations easier to handle. To provide better readability we will omit the arguments where it is obvious. We start from the well-known \textsc{Navier-Stokes} equations
\begin{align}
\label{eq:NS3Dbegin}
    \frac{\partial \bfu}{\partial t}+(\bfu\cdot \nabla)\bfu &= \bff +\nabla\cdot\sigma +\mu_s\Delta\bfu-\nabla p,\\
    \ddiv(\bfu)&= 0,\label{eq:div0}\\
    \bfu &= \bfu_b \text{ on }\partial\hat\Omega,\\
    \bfu(t=0) &=\bfu_0,
\end{align}
where $t\in[0,T]$ is the current time and $T$ the end time of our simulation. We set a domain $\hat\Omega\subset\rr^3$ and an element $\hat\bfx\in\hat\Omega$. We also have $\bfu(t,\hat\bfx)\colon \rr^+\times\hat\Omega\to\rr^3$ as the velocity vector, $\bff(t,\hat\bfx)\colon\rr^+\times\hat\Omega\to\rr^3$ a generic source or sink term and $\sigma(t,\hat\bfx)\colon\rr^+\times\hat\Omega\to\rr^{3\times 3}$ the stress tensor. As usual $\mu_s\in\rr^+$ denotes the solvent viscosity. We will use $p(t,\hat\bfx)\colon\rr^+\times\hat\Omega\to\rr$ as the pressure. Our boundary condition for $\bfu$ is denoted by $\bfu_b$.
The difference between Newtonian and non-Newtonian fluids lies in the definition of $\sigma$. In the UCM or \textsc{Oldroyd-B} model the stress tensor is given by:
\begin{equation}
    \sigma(t,\hat\bfx) = \int_{-\infty}^t-\partial_{t'}\bfB(t,t',\hat\bfx)G(t,t')\D t',
    \label{eq:generalsig}
\end{equation}
where $G\colon\rr^+\times\rr^+\to\rr^+$ is a weight function which decays fast in $t'$. In both physical models which we have discussed in \footnote{reference paragraph} $G$ is given by:
\begin{equation}
    G(t,t')=\mu_p\cdot e^{-(t-t')/\lambda}
    \label{eq:G}
\end{equation} 
where $\mu_p\in\rr^+$ is the polymer viscosity and $\lambda\in\rr$ the relaxation time. Both are fluid- dependent parameters. However, $G$ can be more complex. For example, if $G$ becomes space-dependent, this introduces many new problems. This is why we will stick to \eqref{eq:G} for now.
\par 
The term $\bfB\colon\rr^+\times\rr^+\times\hat\Omega\to\rr^{3\times 3}$ in \eqref{eq:generalsig} is called \emph{Finger tensor} and obeys the differential equation
\begin{equation}
\label{eq:Bfull}
    \partial_t \bfB + (\bfu\cdot\nabla)\bfB-(\nabla \bfu)\bfB-\bfB(\nabla\bfu)^T = \bm{0}.
\end{equation}
We introduce the following initial and boundary conditions
\begin{align}
    \bfB(t',t', \hat\bfx) &\equiv\bm{1}\\
    \bfB(0,t',\hat\bfx) &\equiv\bm{1}
\end{align}
where the last condition implies the assumption that we start the computation stress free. \par
The variable $t'$ which appeared in the last equations is called \enquote{history variable}. We will discuss the impact of this later on.
\par 
Now we will assume that $\bfu, \nabla p $ and $\bff$ are only non-zero in one direction, that is,
\begin{equation}
    \bfu=(0, 0, u)^T,\quad\nabla p= (0,0,\partial_3 p),\quad \bff=(0,0,f)
\end{equation}
Where $u,f,p\colon\rr^+\times\hat\Omega\to\rr$
Because $\ddiv(\bfu)=0$ (see \eqref{eq:div0}) we immediately obtain $\partial_3 u=0$, which leads to
\begin{equation}
    [(\bfu\cdot\nabla)\bfu]_3 = u_1\partial_1 u_3+u_2\partial_2 u_3+u_3\partial_3 u_3 = 0
\end{equation}
and obviously the other components of this term become zero as well. Therefore, we loose the advection term in the \textsc{Navier-Stokes} equations.\\
We will now examine the diffusion term
\begin{equation}
    \Delta \bfu=\begin{pmatrix}
    \Delta u_1\\\Delta u_2\\\Delta u
    \end{pmatrix}=\begin{pmatrix}
    0\\0\\ \partial_1^2u+\partial_2^2u+\partial_3^3u
    \end{pmatrix}=\begin{pmatrix}
    0\\0\\ \partial_1^2u+\partial_2^2 u
    \end{pmatrix}
\end{equation}
If we take a close look at \eqref{eq:NS3Dbegin} we can see that in order to fulfill the equations for $u_1$ and $u_2$ ($0=(\nabla\cdot\sigma)_l,\quad l\in\{1,2\}$), the block with indexes $(i,j):i,j\in\{1,2\}$ of $\sigma$ has to be constant in space. The last row and last column are the same because of symmetry of the tensor. We have to make sure that they are constant in the third spatial direction which is obviously given by our assumptions. 
\par After we treated the \textsc{Navier-Stokes} equations in the last paragraph we will now focus on the governing equation for the Finger tensor. We already saw that only the last row and last column are of interest for us. Let's look at \eqref{eq:Bfull} in index notation, which reads
\begin{equation}
\label{eq:Bindex}
    \partial_t \bm{B}_{i,j}+\sum_{k=1}^3\bm{u}_k\partial_k \bm{B}_{i,j}-\sum_{k=1}^3\partial_k\bm{u}_i\bm{B}_{k,j}-\sum_{k=1}^3\bm{B}_{i,k}\partial_k\bm{u}_j=0
\end{equation}
We can now put our assumptions for $\bfu$ into it. Additionally $\partial_3\bfB_{3,j}$ should be $0$. We will just focus on the last row for now. This leads to
\begin{equation}
    \partial_t \bm{B}_{3,j} -\sum_{k=1}^2\partial_ku\bm{B}_{k,j}-\left.\begin{cases}
    0 &, j=1,2\\ \sum_{k=1}^2 \bm{B}_{3,k}\partial_ku &, j=3
    \end{cases}\right\} =0.
\end{equation}
If we revisit \eqref{eq:Bindex}, we observe that
\begin{equation}
     \partial_t \bm{B}_{i,j}=-u(\partial_3 \bm{B}_{i,j})~\forall(i,j)\in\{1,2\}^2
\end{equation}
This shows that if we choose this block constant in the beginning as required by our assumptions it will not change over time. In our case we will choose the identity matrix for this. These observations also justify that we only care about the last row and ignore the other ones. If we define $\bfb_j:=\bfB_{3,j}$ we obtain the equation
\begin{equation}
   \partial_t \bm{b}=\begin{pmatrix}
   \partial_1u\\ \partial_2 u \\ 2\left((\partial_1 u)\bm{b}_1+(\partial_2 u)\bm{b}_2\right)
   \end{pmatrix}.
\end{equation}
We have stated above that $\partial_3\bfb_3=0$ . This directly leads to $\partial_3\sigma_{3,3}=0$. Now, taking a good look at \eqref{eq:NS3Dbegin}, we see that $\sigma_{3,3}$ only contributes with its third spatial derivative. So it vanishes from the equations completely. We redefine
\begin{equation}
\partial_t\bfb=
    \begin{pmatrix}
    \partial_1 u\\\partial_2 u
    \end{pmatrix}
\end{equation}
Furthermore, we define $\bfs:=\sigma_{3,j}$. Consequentially, we obtain the governing equation
\begin{equation}
    \bfs(t,\hat\bfx) =\int_{-\infty}^t-\partial_{t'}\bfb(t,t',\hat\bfx)G(t,t')\D t'.
\end{equation}
We can now define our new system of equations
\begin{align}
\partial_t u(t,\bfx) &= -\nabla p + f +\nabla\cdot \bfs+\mu_s\Delta u,\\
\bfs(t,\bfx) &=\int_{-\infty}^t-\partial_{t'}\bfb(t,t',\bfx)G(t,t')\D t',\label{eq:s2D}\\
\partial_t\bfb(t,t',\bfx)&=
\begin{pmatrix}
\partial_1 u(t,\bfx)\\\partial_2 u(t,\bfx)
\end{pmatrix},
\end{align}
where $\Omega\subset\rr^2$ is our domain with an element $\bfx$. We let $u(t,\bfx), p(t,\bfx), f(t,\bfx)\colon\rr^+\times\rr^2\to\rr$ denote the third component of the velocity, th pressure and a generic source/sink term respectively. $\Delta$ now stands for the two-dimensional \textsc{Laplace}-operator. To define the stress and finger tensor we use $\bfs(t,\bfx)$ and  $\bfb(t,\bfx)\colon\rr^+\times\rr^2\to\rr^2$ respectively.

By using our assumptions we now deducted a true two-dimensional problem. This can be interpreted as a slice orthogonally to the fluid's flowing direction. During this deduction we silently eliminated the nonlinearity in $\bfb_3$ because it is not relevant for the equations anymore. This makes work much easier as we will see later on.
\subsection{\textsc{Laplace} transformation}
The next step that is usually taken is the introduction of an \enquote{age} variable $\tau=t-t'$. This is sensible because if we take a look at \eqref{eq:s2D}, we see that only the times $t'$ close to $t$ are relevant (recall that $G$ is exponentially declining the larger the difference between $t$ and $t'$). The further back in history $t'$ gets, the less contribution it makes. By putting this transformation into our equations we obtain
\begin{equation}
    \bfs(t,\bfx)=\int_0^\infty\partial_\tau\bfb(t,t-\tau,\bfx)G(t,t-\tau)\D\tau.\label{eq:sage}
\end{equation}
Using the chain rule the governing equation for the Finger tensor transforms to  
\begin{equation}
    \partial_t \bfb +\partial_\tau\bfb = \begin{pmatrix}
    \partial_1 u(t,\bfx)\\\partial_2 u(t,\bfx)
    \end{pmatrix},
\end{equation}
with the corresponding boundary and initial conditions
\begin{align}
    \bfb(t,t) &= \bm{0}\\
    \bfb(t,0) &= \bm{0}
\end{align}
On first glance this appears to have made our situation worse because we now have two time derivatives in the equation. Luckily there is a tool that can help us with that: the \textsc{Laplace} transformation. We define
\begin{equation}
L_{\bfb}(t,\bfx,s) := \mathcal{L}\{\bfb\}(t,\bfx,s) \int_0^\infty\bfb(t,t-\tau,\bfx)e^{-s\tau}\D\tau
\end{equation}
as the \textsc{Laplace} transform of $\bfb$, where $s\in\cc$ is a transformation parameter. 
Because the integral and the differential operator are linear we observe that this results in
\begin{equation}
    \partial_tL_{\bfb}(t,\bfx,s) + L_{\partial_\tau\bfb}(t,\bfx,s) = \begin{pmatrix}
    \partial_1 u(t,\bfx)\\\partial_2 u(t,\bfx)
    \end{pmatrix}.
\end{equation}
However, because $\partial_\tau$ is not independent of $\tau$, we have use integration by parts to transform it to
\begin{align}
    L_{\partial_\tau\bfb}(t,\bfx,s) &= \int_0^\infty\partial_\tau\bfb(t,t-\tau,\bfx)e^{-s\tau}\D\tau\\ &=\lim_{r\to\infty}\bfb(t,t-r,\bfx)e^{-sr}-\bfb(t,t,\bfx)e^{-s\cdot 0}+s\int_0^\infty\bfb(t,t-\tau)e^{-s\tau}\D\tau\\
    \intertext{the first term vanishes because $e$ is dominant so it remains that}
    &= -\bfb(t,t,\bfx) +sL_{\bfb}(t,\bfx, s)\\
    &= sL_{\bfb}(t,\bfx,s).
\end{align}
We have successfully eliminated the time derivative in $\tau$ direction. The governing equation for the \textsc{Laplace} transform of the finger tensor is given by
\begin{equation}
(\partial_t +s)L_{\bfb}(t,\bfx,s) = \begin{pmatrix}
\partial_1 u(t,\bfx)\\\partial_2 u(t,\bfx)
\end{pmatrix}.
\end{equation}
The corresponding initial condition is 
\begin{equation}
    L_{\bfb}(0,\bfx,s) = \bm{0}
\end{equation} 
As a next step we will introduce the conformation tensor, which non-dimensionalizes $L_{\bfb}$ . It is defined as 
\begin{equation}
    \bfC_s(t,\bfx) := sL_{\bfb}(t,\bfx,s)
\end{equation}
and has the governing equation
\begin{equation}
    \partial_t\bfC_s(t,\bfx)+s\bfC_s(t,\bfx) =\begin{pmatrix}
    s\partial_1u\\s\partial_2 u
    \end{pmatrix}.
\end{equation}
This transformation worked very well and eliminated many problems but the question remains: How do we retrieve $\bfs$?
Let's recall \eqref{eq:sage}.
Because we only have a finite history and start stress free we can cut the integral off at $t$
\begin{equation}
    \bfs(t,\bfx)=\int_0^t\partial_\tau\bfb(t,t-\tau,\bfx)G(t,t-\tau)\D\tau.\label{eq:sagered}
\end{equation}
If we assume that the \textsc{Laplace} transform of $G$ is well-defined and allow that $G$ is space dependent we can set
\begin{equation}
    g(\bfx,t,s) =\int_0^\infty G(t,t',\bfx)e^{-st'}\D t'.
\end{equation}
Then $\bfs$ is simply given by an inverse Laplace transform so specifically
\begin{equation}
    \bfs(t,\bfx) =\mathcal{L}^{-1}\{\bfC_s(t,\bfx)g(t,\bfx,s)\}(t,\bfx)
\end{equation}
At first glance one may we tempted to be satisfied with this. However, numerically it is unclear how this would work. One of the nice properties of this approach is that if we choose 
\begin{equation}
    G(t,t-\tau)= \mu_p e^{-\tau/\lambda},\quad s=\frac{1}{\lambda},
\end{equation}
where $\lambda$ is the material specific relaxation time, we can retrieve the \textsc{Maxwell} / \textsc{Oldroyd}-B model as follows
\begin{align}
    \bfs(t,\bfx)&=\mu_p\int_0^t\partial_\tau\bfb(t,t-\tau,\bfx)e^{-\tau/\lambda}\D\tau\\
    \intertext{using integration by parts again}
    &=\mu_p\bfb(t,0)e^{-t/\lambda}-\mu_p\bfb(t,t)e^{-0/\lambda}+\frac{\mu_p}{\lambda}\int_0^t\bfb(t,t-\tau)e^{-\tau/\lambda}\D\tau\\
    \intertext{if we use the boundary and initial conditions and extend the integral to $\infty$ we get}
    &=\frac{\mu_p}{\lambda}\int_0^\infty\bfb(t,t-\tau)e^{-\tau/\lambda}\D\tau\\
    &=\frac{\mu_p}{\lambda}L_{\bfb}(t,\bfx,1/\lambda) = \mu_p\bfC_{1/\lambda}
\end{align}
So using this assumptions we removed the integral term completely! In this example the stress tensor is just the scaled conformation tensor. 
\section{Discretization and building the numerical scheme}
The next step will be to build a numerical scheme to solve these equations. For this we firstly need the weak formulation then we will discuss the discretizations in space and time.
\subsection{Weak formulation}
\subsection{Spatial discretization}
\subsection{Time discretization}
\section{Numerical results}
\section{Conclusion}

\addcontentsline{toc}{section}{List of figures}
\setcounter{lofdepth}{2}
\listoffigures
\newpage
\addcontentsline{toc}{section}{References}
\bibliographystyle{amsplain}
\bibliography{References}{}
\newpage
\section*{Eigenständigkeitserklärung}
Hiermit versichere ich, Nils Dornbusch, an Eides statt, dass ich die vorliegende Arbeit selbstständig und ohne die Benutzung anderer als der angegebenen Hilfsmittel angefertigt habe. Alle Stellen, die wörtlich oder sinngemäß aus veröffentlichten und nicht veröffentlichten Schriften entnommen wurden, sind als solche kenntlich gemacht. Die Arbeit ist in gleicher oder ähnlicher Form oder auszugsweise im Rahmen einer anderen Prüfung noch nicht vorgelegt worden. Ich versichere, dass die eingereichte elektronische Fassung der eingereichten Druckfassung vollständig entspricht.
\\[\bigskipamount]
Köln, \today
\\[2\bigskipamount]
Nils Dornbusch
\end{document}
